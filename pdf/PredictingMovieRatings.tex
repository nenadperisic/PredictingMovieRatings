% !TEX encoding = UTF-8 Unicode
\documentclass[a4paper]{article}

\usepackage{color}
\usepackage{url}
\usepackage[T2A]{fontenc} % enable Cyrillic fonts
\usepackage[utf8]{inputenc} % make weird characters work
\usepackage{graphicx}
\usepackage{listings}
\usepackage{multicol}
\usepackage{geometry}
\usepackage{lipsum}

%\usepackage[nottoc]{tocbibind}

\usepackage[english,serbian]{babel}
%\usepackage[english,serbianc]{babel} %ukljuciti babel sa ovim opcijama, umesto gornjim, ukoliko se koristi cirilica

\usepackage[unicode]{hyperref}
\hypersetup{colorlinks,citecolor=green,filecolor=green,linkcolor=blue,urlcolor=blue}

%\newtheorem{primer}{Пример}[section] %ćirilični primer
\newtheorem{primer}{Primer}[section]

\definecolor{mygreen}{rgb}{0,0.6,0}
\definecolor{mygray}{rgb}{0.5,0.5,0.5}
\definecolor{mymauve}{rgb}{0.58,0,0.82}
\definecolor{greyish}{rgb}{0.97,0.97,0.97}


\lstset{ 
  backgroundcolor=\color{greyish},   % choose the background color; you must add \usepackage{color} or \usepackage{xcolor}; should come as last argument
  basicstyle=\scriptsize\ttfamily,        % the size of the fonts that are used for the code
  breakatwhitespace=false,         % sets if automatic breaks should only happen at whitespace
  breaklines=true,                 % sets automatic line breaking
  captionpos=b,                    % sets the caption-position to bottom
  commentstyle=\color{mygreen},    % comment style
  deletekeywords={...},            % if you want to delete keywords from the given language
  escapeinside={\%*}{*)},          % if you want to add LaTeX within your code
  extendedchars=true,              % lets you use non-ASCII characters; for 8-bits encodings only, does not work with UTF-8
  firstnumber=1,                   % start line enumeration with line 1000
  frame=single,	                   % adds a frame around the code
  keepspaces=true,                 % keeps spaces in text, useful for keeping indentation of code (possibly needs columns=flexible)
  keywordstyle=\color{blue},       % keyword style
  language=Python,                 % the language of the code
  morekeywords={*,...},            % if you want to add more keywords to the set
  numbers=left,                    % where to put the line-numbers; possible values are (none, left, right)
  numbersep=5pt,                   % how far the line-numbers are from the code
  numberstyle=\tiny\color{mygray}, % the style that is used for the line-numbers
  rulecolor=\color{black},         % if not set, the frame-color may be changed on line-breaks within not-black text (e.g. comments (green here))
  showspaces=false,                % show spaces everywhere adding particular underscores; it overrides 'showstringspaces'
  showstringspaces=false,          % underline spaces within strings only
  showtabs=false,                  % show tabs within strings adding particular underscores
  stepnumber=2,                    % the step between two line-numbers. If it's 1, each line will be numbered
  stringstyle=\color{mymauve},     % string literal style
  tabsize=2,	                   % sets default tabsize to 2 spaces
  title=\lstname                   % show the filename of files included with \lstinputlisting; also try caption instead of title
}

\definecolor{dkgreen}{rgb}{0,0.6,0}
\definecolor{gray}{rgb}{0.5,0.5,0.5}
\definecolor{mauve}{rgb}{0.58,0,0.82}


\begin{document}

\title{\textbf{Predviđanje filmskih rejtinga}\\
\small{Projekat u okviru kursa Mašinsko učenje}}

\author{Nenad Perišić\\perisicnenad96@gmail.com\\}

\medskip

\newpage

\maketitle

%\begin{figure}[hb!]
%\begin{center}
%\small{Matematički fakultet \\ u Beogradu\\}
%\includegraphics[scale=0.5]{matf_logo.png}
%\end{center}
%\end{figure}

%\vfill
\vspace{2cm}
\begin{figure}[b!]
\begin{center}
\includegraphics[scale=0.67]{matf_logo.png} \\
\small{Matematički fakultet \\ u Beogradu}
\end{center}
\end{figure}


\pagebreak

\tableofcontents

\newpage

\section{Uvod}
\label{sec:uvod}
U ovom radu ćemo pokušati da damo odgovor na pitanje da li je moguće predvideti rejting filma i proceniti njegovu uspešnost pre nego što on bude pušten u bioskopima. Ovo ćemo uraditi analiziranjem skupova multimodalnih podataka sa raznim atributima (svojstvima) kao što su režiser tog filma, glumačka ekipa, kompanija koja je radila produkciju filma, opis filma, žanr filma, poster filma itd. Mi ćemo se osvrnuti na predviđanje rejtinga na osnovu tekstualnih atributa i na njima primenjivati algoritme linearne regresije, grebene regresije, slučajne šume i neuronske mreže.


\section{Skup podataka}
\label{sec:skupPodataka}
Koristićemo dva skupa podataka, "The movie dataset" \cite{movieDataset} i "Movie genre from it's poster dataset" \cite{movieGenres}, oba dostupni na Kaggle \cite{kaggle} veb sajtu. Oni sadrže metapodatke za 45000 filmova koji su objavljeni pre jula 2017. godine. Podaci se sastoje od glumačke ekipe (cast), celokupne ekipe (crew), ključnih reči radnje, budžeta, prihoda, postera, datuma objavljivanja, jezika, produkcijskih kuća, država, broja glasova i njihovog proseka.\\

Skup podataka je organizovan na sledeći način:
\begin{itemize}
	\item \textit{movies\_metadata.csv} - glavna datoteka metapodataka o filmovima
	\item \textit{keywords.csv} - sadrži ključne reči filmskog zapleta dostupog u formi JSON objekta
	\item \textit{credits.csv} - sastoji se od glumačke ekipe takođe u formi JSON objekta
	\item \textit{links.csv} - sadrži TMDB i IMDB id-ove svih filmova
	\item \textit{ratings\_small.csv} - sadrži rejtinge od 100 000 rejtinga koji su prikupljeni od 700 korisnika na 9000 filmova
	\item \textit{movie\_genre.csv} - sadrži žanrove filmova koje ćemo spajati preko imdb\_id-a, a koja takođe sadrži i link do postera filma
\end{itemize}


\section{Preprocesiranje}
\label{sec:preprocesiranje}

Za ulazne parametre posmatraćemo atribute cast, crew, synopses, genre i runtime, a IMDB Score ćemo uzeti za cilj predvidjanja. Iz kolona cast i crew ćemo da izdvojimo dva glavna glumca, režisera i produkcijsku kuću za svaki film ali tako da su se pojavljivali u barem 5 filmova kako nam se ne bi narušilo predviđanje ukoliko je neko imao samo jedan film u svojoj karijeri. Takođe, zarad boljih rezultata možemo uzeti samo filmove koji su prvi put prikazani posle 1970. godine i čiji je originalni jezik engleski.


Podelu skupa podataka na trening, test i validacioni skup ćemo izvršiti u odnosu 70\%:15\%:15\%.\\




\section{Modeli}
\label{sec:modeli}




\pagebreak

\section{Zaključak}
\label{sec:zakljucak}
zakljucak


\pagebreak

\addcontentsline{toc}{section}{Literatura}
\appendix
\bibliography{reference}
\bibliographystyle{ieeetr}
\appendix

\end{document}
